\documentclass[10pt,a4paper]{article}
\usepackage[latin1]{inputenc}
\usepackage{amsmath}
\usepackage{amsfonts}
\usepackage{amssymb}
\usepackage{graphicx}
\title{Algorithms and Data Structures Summative - Question 5}
\begin{document}
	\underline{a:}\\
	$T(n) = 9T(n/3) + n^2$ so $a=9$, $b=3$ and $f(n) = n^2$\\
	Is $n^2 = \Theta(n^{\log_3(9)})$? [Rule 2]\\
	Is $n^2 = \Theta(n^{2})$? True.\\
	$\therefore T(n) = \Theta(n^2\log n)$\\
	
	\underline{b:}\\
	$T(n) = 4T(n/2) + 100n$ so $a=4$, $b=2$ and $f(n) = 100n$\\
	Is $100n = O(n^{\log_2(4) - \epsilon})$? [Rule 1]\\
	Is $100n = O(n^{2 - \epsilon)}$? True.\\
	$\therefore T(n) = \Theta(n^2)$\\
	
	\underline{c:}\\
	$T(n) = 2^nT(n/2) + n^3$ so $a=2^n$, $b=2$ and $f(n) = n^3$\\
	$a$ is not constant $\therefore$ Master Theorem cannot be applied.\\
	
	\underline{d:}\\
	$T(n) = 3T(n/3) + cn$ so $a=3$, $b=3$ and $f(n) = cn$\\
	Assuming c is constant term then Master Theorem can be applied.\\
	Is $cn = \Theta(n^{\log_3(3)})$? [Rule 2]\\
	Is $cn = \Theta(n)$? True.\\
	$\therefore T(n) = \Theta(n\log n)$\\
	
	\underline{e:}\\
	$T(n) = 0.99T(n/7) + 1/(n^2)$ so $a=0.99$, $b=7$ and $f(n) = 1/(n^2) = n^{-2}$\\
	Is $n^{-2} = \Omega(n^{\log_7(0.99) + \epsilon})$? [Rule 3]\\
	Is $n^{-2} \approx \Omega(n^{-0.005 + \epsilon})$? False.\\
	\\
	Is $n^{-2} = O(n^{\log_7(0.99) - \epsilon})$? [Rule 1] True.\\
	$\therefore T(n) = \Theta(n^{log_7(0.99)})$
\end{document}